\load[ctustyle3]
\worktype[B/EN]
\faculty{F8}
\department{Katedra Počítačových Systémů}
\title{NPU/AI Accelerator support for PikeOS real-time operating system}
\author{Michal Žáček}
\abstractEN{D}
\abstractCZ{D}
\declaration{D}
\date{2025-05-16}
\draft

\def\OpenVX{OpenVX™}
\def\VeriSilicon{VeriSilicon}
\def\NXP{NXP}
\def\textemdash{-}

\totf
\makefront

\chap Introduction

Our computing devices have come a long way in terms of speed
and of course in terms of operating complexity.
This complexity can hold back very general purpose processing units
in performance when it comes to highly specialized tasks.
For a long time GPUs were used where a large number
of specialized identical operations needed to be performed on
a vast array of differing data inputs,
as their specialized design for this use-case made them incomparably faster than
using a common general purpose CPU.
Though most often associated with their namesake of graphics,
they excel at many applications and have become a
valuable tool when working with neural networks.

Recently even more specialized hardware has become available
to meet the rising demand in having access to these technologies
in more portable devices or for use in embedded situations.
Neural Processing Units\fnote{
Historically also called Versatile/Tensor/Intelligence Processing Units (VPU/TPU/IPU),
however for simplicity's sake we will use the term NPU and only deviate if a specific
function or product uses one of the alternatives
},
are becoming more common,
however these don't often come with universal drivers
and the most widespread standard today has no freely adoptable implementation.

\sec Goals

This thesis will initially lay out the defined and known terms used in the context of neural nets, such as hardware, alternate names of known concepts or file formats.
Following this we shall go through the existing solutions implementing both the communication with actually hardware in the form of modules followed by frontend APIs for NPU work.
We will then go through the advantages that this hardware should in theory provide followed by a practical test of the actual results on real hardware.
These will be packaged into demonstration packages than may be run to show off and/or verify these results.
Finally, we discuss what this means for PikeOS integration of NPU support.

\sec NPUs (Neural Processing Units)

Neural Processing Units (further shortened to NPU) are in general a type of hardware accelerator which is
optimized to efficiently handle Machine Learning workloads.
Our NPU in question is composed of a frontend that is used for
communication between the NPU and the rest of the board,
a parallel processing unit,
a neural network engine
along with a backing storage.
The device works with 4-element vectors of 16/32-bit IEEE floating-point or 8/16/32-bit integers either signed or unsigned.
The device implements the \OpenVX~hardware API
along with some extensions and
\VeriSilicon~provides two separate official libraries through which to call this API.
The open-source Tensor Interface Module\cite[VerisiliconTim2025]
(further shortened to TIM-VX) which contains C++ bindings,
and Ovxlib which contains C bindings.
TIM-XV calls into one of two SDKs, the \VeriSilicon~Unified \OpenVX~SDK
supporting both compiling and running using either the GPU and NPU units,
and the
(as far as I can tell proprietary,
since the traces of this SDK online
are close to nonexistent)
VIP-Lite SDK that only contains the runtime for the NPU.
TIM-VX is used as the target for third-party frameworks targetting the board's NPU.
Of these we used specifically the former VeriSilicon SDK
also often listed as Vivante SDK or directly
\code{imx-gpu-viv}\fnote{The name is GPU, however it contains shared drivers for both NPU and GPU.}
in recipes,
as this was the one used by all of the NXP recipes.

The SDK includes the following libraries:
\begitems
* \code{libArchModelSw.so*}
* \code{libCLC.so}
* \code{libGAL.so*}
* \code{libNNArchPerf.so*}
* \code{libOpenVX.so*}
* \code{libOpenVXC.so}
* \code{libOpenVXU.so}
* \code{libVSC.so*}
\enditems

Of these only the ones marked with \code{*} are used by \OpenVX{}.

\glos {TIM-VX}{Tensor Interface Module for OpenVX}

\chap File Formats

\sec Model File Properties

\secc Model conversion

Apparently there are ways to compile a model for a specific NPU,
even multiple models so they efficiently live together
in memory\cite[CoralEdgeTpuCompiler].\rfc{BIG TODO}

TensorFlow is used in preparation and then compiled to a lighter LiteRT format\cite[LitertOverview].

\secc Quantization

Since the NPU itself is built for
all of 8/16/32-bit wide floats \cite[IMxMachineLeLfRe2024]
and different sizes of integers,
we may wish to lessen the load and speed-up the inference
by opting to sacrifice the precision of the result
and instead of running the entire process
with F32, we use INT8,
according to \cite[PyTorchQuant]
this may lead to the operations being implemented 2 to 4
times faster with integers.

\sec Keras / HDF5

Keras is a library released in 2010
as part of a research effort,
originally at Alphabet Inc\cite[KerasWikiped].
Its relation to TensorFlow will be apparent later.
Keras' currently used filetype \code{.keras} is a wrapper format consisting of
a ZIP-archive containing an \code{.h5} file
holding the layers and weights of a model,
and some \code{json} metadata about Keras' configuration,
the creation date, version information, etc.

HDF stands for Hierarchical Data Format,
it contains a POSIX-filesystem style hierarchy of groups
akin to directories,
even supporting hard and soft links.
Datasets take the role of files and contain
tensors with associated shapes, dtypes and
can be read and written to treating them as numpy arrays\cite[HDF5PY,HDF5FGGS].

\begtt\hisyntax{py}
import h5py

data = h5py.File("model.weights.h5")

# All valid
data['/layers/flatten/vars']
data['layers/flatten/vars']
subdata = data['layers']['flatten']
subdata['vars']

data['/layers/dense/vars/0']
#=> <HDF5 dataset "0": shape (784, 128), type "<f4">

data['/layers/dense/vars/0'][0]
#=> array([ 0.01777279,  0.07076738,
           -0.01541622,  0.06383523,
           -0.05744237,  0.06602553,
           ...])

\endtt

As compression is done by Keras in-memory
so saving and loading are transparent
to wether one is working with a ZIP archive
or a directory of files.

\sec Safetensors

\midinsert \clabel[safetensordef]{SafeTensor format structure}
\ctable{ll}{
	Size & Content \crl
	8 bytes & size of header \cr
	N bytes & header \cr
	rest & byte-buffer (little-endian)
}
\caption/t SafeTensor format structure
\endinsert

Header\urlnote{https://github.com/huggingface/safetensors}:
JSON UTF-8 string, each top-level key is a tensor name the value of which contains a dictionary specifying the dtype, shape and data_offsets.

A schema\urlnote{https://github.com/huggingface/safetensors/blob/main/docs/safetensors.schema.json} is available.
It allows partial loading

\sec Pickle

	{\bf\Red This is dangerous to use.}
Since pickling allows arbitrary instances of classes,
malicious code may be inserted as well into any model downloaded.

An example taken from \cite[ExploitingPythHamann2020].
\verbinput \hisyntax{python} (-) pickle_exploit.py

\sec Block map \code{.bmap}

Relates to Yocto project's (formerly Intel's) \code{bmaptool} from \cite[YoctoBmap],
which is used to flash data similarly to \code{dd}.
Unlike \code{dd} this tool verifies the integrity
of flashed data, supports more complex arrangements
such as sourcing the image from a remote server,
supports sparse definitions
and includes protections from accidentally destroying
data on disks that seem like regular mounted block devices.

\sec System Package Data Exchange \code{.spdx}

A Bill of Materials of all the included
packages used to build an image including all the versions
and metadata.
\urlnote{https://en.wikipedia.org/wiki/Software_Package_Data_Exchange}

\sec OpenEmbedded Image Creator \code{.wic}

Should your device require
multiple partitions.\urlnote{https://docs.yoctoproject.org/dev/dev-manual/wic.html}
Use if \code{bmap} isn't supported as it doesn't support
sparsness, etc.\urlnote{citation needed}

\secc OpenEmbedded kickstart file \code{.wks}

Contains build commands for the \code{wic} command.
\urlnote{https://docs.yoctoproject.org/dev/dev-manual/wic.html}

\secc Flattened Device Tree \code{.fdt} / Device Tree Source \code{.dts???}

\code{.dts} describes hardware.\urlnote{https://wiki.freebsd.org/FlattenedDeviceTree\#Device\_tree\_source\_.28DTS.29}
Afterwards compiled into \code{.dtb}.

\chap TIM-VX

\midinsert
\picw=\hsize
\cinspic{timvx_overview.pdf}
\caption/f A diagram showing TIM-VX's role in running models on NPUs\cite[VerisiliconTim2025].
\endinsert

\chap Setting up the environment

Our testing environment consists of a Yocto distribution,
running on our afformentioned chip.

\code{meta-overlay} will further be used to refer
to our custom layer for the purposes
of this thesis.

We use both the BSP and related layers directly from the TQ website,
followed by the \NXP{} meta-imx layer,
cloned via the \code{repo}\cite[GoRepo] utility from their manifest
repository \cite[GithubNxpIm].

Repo is a utility developed by Alphabet
to work with multiple, completely separate
as far as Git is concerned,
repositories and treat them
as Git would treat submodules.
The repositories are initialized from
an Extensible Markup Language (XML)
file hosted in an arbitrary repository,
which contains a set of remotes and
"projects" (repositories)
that will be fetched into the final checkout.

\glos {XML}{Extensible Markup Language}

\begtt
repo init \
  -u https://github.com/nxp-imx/imx-manifest.git \
  -b imx-linux-scarthgap \
  -m imx-6.6.52-2.2.0.xml
\endtt

Afterwards a bit of guesswork was required to get
all the compatible versions of all the layers and packages.
A combination of some layers from
the original BSP and
the NXP-IMX manifest was used in the end.
Mostly due to mismatches in \code{gstreamer} versions
as meta-imx requires a version not below 1.24.0,
but the bsp provides 1.22.5.
The result of this is found in the Appendix \ref[buildconf].

Following this we configure a build directory
with the Machine: \code{tqma8mpxl-mba8mpx}
and Distro: \code{fsl-imx-wayland}.

\begtt
source setup-environment tqma8mpxl_build
\endtt

Next we need an environment with all the programs that bitbake expects
and needs to function.
In our case we opted for an Ubuntu:22.04 Docker container,
with some extra packages installed as determined by consulting \cite[YoctoPackages]:

\begtt
FROM ubuntu:22.04
RUN apt -o APT::Sandbox::User=root update
RUN DEBIAN_FRONTEND=noninteractive TZ=Etc/UTC \
      apt -o APT::Sandbox::User=root \
      install -y gawk wget git diffstat unzip texinfo gcc \
      build-essential chrpath socat cpio python3 python3-pip \
      python3-pexpect xz-utils debianutils iputils-ping \
      python3-git python3-jinja2 python3-subunit zstd \
      liblz4-tool file locales libacl1
RUN locale-gen en_US.UTF-8
\endtt

After building we may enter the environment:

\begtt
docker run --rm -it \
  -v .../scarthgap.TQ.ARM.BSP.0001:/src \
  --userns=keep-id bitbake-env
\endtt

And we may now build the project:

\begtt
bitbake imx-image-full
\endtt

The output will be available under:

\begtt
./tmp/work/imx8mpevk-poky-linux/imx-image-full/...
  .../1.0/deploy-imx-image-full-image-complete
\endtt

This may take many hours.
Once successful, flashing is then done with the command:
\begtt
sudo uuu -v -b sd_all \
  ./images/imx-boot-tqma8mpxl-mba8mpxl-mfgtool.bin-flash_spl_uboot \
  ./images/imx-image-full-tqma8mpxl-mba8mpxl.rootfs.wic
\endtt

\sec Kernel Tests

The VSOCK Makefile has no install goal, yet the recipe tries to call it,
so we just disable that part of the recipe with the following:

\begtt
PACKAGECONFIG:kernel-tools = ""
\endtt

\sec ONNXRuntime

We change the original ONNXRuntime recipe's
source from NXP to the official repo
as it now supports our NPU through the use of the
TIM-VX library:

\begtt
ONNXRUNTIME_SRC ?= "gitsm://github.com/microsoft/onnxruntime.git"
SRC_URI = "${ONNXRUNTIME_SRC};nobranch=1;protocol=https
# Rel-1.21.0
SRCREV = "e0b66cad282043d4377cea5269083f17771b6dfc"
\endtt

ONNXRuntime's configuration script can't by default find our
installation of TIM-VX, so we must assist it a bit,
in addition to adding it to \code{(R)DEPENDS}:

\begtt
DEPENDS = "libpng zlib tim-vx tvm"
RDEPENDS:${PN} = "tim-vx tvm"

do_configure:prepend () {
    export TIM_VX_INSTALL="/usr"
}
\endtt

Next we must also actually enable the use of this library:

\begtt
EXTRA_OECMAKE += "\
    -Donnxruntime_USE_VSINPU=ON \
    -Donnxruntime_USE_TVM=ON \
"
\endtt

As of writing the main \code{1.21.0} version of ONNXRuntime
is the first to support VSINPU, however
the release version had broken support for
targets with no fp16 support,
therefore a pr's commit with the fix needs to be used.

\code{GitHub - Issue: 23957, PR: 23978}

It is very fast on track to be merged into main,
but now we apply these few commits with a patch.

\begtt
SRC_URI:append = " file://fajin-corp_gh_pr_23978.patch"
\endtt

Another patch is required though as the function

\code{VSINPUExecutionprovider::GetCapability}

in the file \code{vsinpu_execution_provider.cc}
calls a logger, yet omits to declare one so
we must add it manually.

\begtt
const auto& logger = *GetLogger();
\endtt

The call to \code{save_build_and_package_info}
in \code{setup.py},
causes an error.
It is only a buildinfo log,
so simply removing it creates a warning
when importing the module,
however without hindering further use.

The patch is included with the work.

\begtt
SRC_URI:append = " file://fix_logger.patch"
\endtt

\sec TIM-VX

This was simply needed to be updated to version 1.2.22,
from the official VeriSilicon repo \ref[tim_bb].

\midinsert
\label[tim_bb]
\begtt
SRC_URI = "${TIM_VX_SRC};nobranch=1"
TIM_VX_SRC ?= "git://github.com/VeriSilicon/TIM-VX.git;protocol=https"
SRCREV = "8494275d7608942aa584c9c13bd5e2d77be9906c"
\endtt
\caption/f TIM-VX's new version bb recipe
\endinsert

\rfc{I don't remember why??? Rerun bitbake without update}

\sec OpenCV

The build recipe for OpenCV is patchable,
we add TIM-VX to dependencies,
enable it in OpenCV and again
point the configuration script at our installation
directory with the file \code{opencv_\%.bbappend}
containing \ref[opencv_bb].

\midinsert
\label[opencv_bb]
\begtt
EXTRA_OECMAKE:append = "\
    -D BUILD_opencv_gapi=OFF \
    -D WITH_TIMVX=ON \
    -D TIMVX_INSTALL_DIR=/usr/lib \
"

DEPENDS:append = " tim-vx"
RDEPENDS:${PN}:append = " tim-vx"

INSANE_SKIP:${PN}-dbg += "libdir file-rdeps"
INSANE_SKIP:${PN} += "buildpaths"
\endtt
\caption/f OpenCV recipe to build with external TIM-VX
\endinsert

Compiling with external TIM-VX lib like this,
which is strongly advised against\cite[OpenCVTimVxBackend],
yields a segfaulting library which can be seen in \ref[cvvx_seg].

\medinsert
\label[cvvx_seg]
\begtt\hisyntax{python}
>>> cv.setUseOpenVX(True)
Traceback (most recent call last):
  File "<stdin>", line 1, in <module>
cv2.error: OpenCV(4.10.0) .../src/ovx.cpp:101:
  error: (-215:Assertion failed)
    !flag && "OpenVX support isn't enabled at compile time"
    in function 'setUseOpenVX'

>>> cv.useOpenVX()
False

# Segfaults
self._model = cv.FaceDetectorYN.create(...)
\endtt
\caption/f Trying to enable OpenVX in OpenCV with external library
\endinsert

Trying to compile TIM-VX directly in OpenCV requires us
to add Fortran to our toolchain due to a new dependence on
\code{lapack} lest we encounter error \ref[fortran_missing_err].
We also must re-enable OpenCV downloads since the default
recipe tries to prevent random downloads during configuration,
by caching the downloads themselves.
This can be seen in our new recipe \ref[cvvx].

\midinsert
\label[fortran_missing_err]
\begtt
libgfortran was skipped:
  libgfortran needs fortran support to be enabled in the compiler
\endtt
\caption/f Fortran missing from Toolchain error
\endinsert

\midinsert
\label[cvvx]
\begtt
EXTRA_OECMAKE:append = "\
    -D BUILD_opencv_gapi=OFF \
    -D WITH_TIMVX=ON \
    -D OPENCV_ALLOW_DOWNLOADS=ON \
"

DEPENDS:append = " tim-vx lapack"
RDEPENDS:${PN}:append = " tim-vx lapack"
\endtt
\caption/f OpenCV recipe to build with internal TIM-VX
\endinsert

Enabling Fortran breaks the \code{tpm2-tss-engine}
and \code{isp-imx-dev} packages
that is requires by \code{packagegroup-imx-ml}
so we try disabling what is wrong:

\begtt
RDEPENDS:packagegroup-imx-ml:remove = " tpm2-tss-engine"
INSANE_SKIP:isp-imx-dev += "dev-elf"
\endtt

This compiles an image,
however OpenCV still throws an error when calling
\code{cv.setUserOpenVX(True)} claiming
it is not compiled with the given support
and when run anyways,
the speeds are identical for both
set backends,
however the speed seems suspiciously high
(in magnitude of hundreds of microseconds).

\chap The Graph Workflow

\sec Python Subclasses modelling Tensor functions

All frameworks have Python APIs,
and all seem to use a similar form of abstraction.
They declare some class in the case of both PyTorch and ONNX
this could be \code{torch.nn.Module}\cite[ModulePytorc].
A model is then defined as a class that inherits from this root,
and overrides one or more specific methods.

Once we instantiate such a class
we receive an object that takes
tensors and returns results.
They can be freely composed into graphs
of arbitrarily complex modules containing modules
which all get compiled to a single graph.

For efficiency's and compatibility's sake these functions are
later compiled into other forms
especially when serialized into the form of any model file.

In the case of ONNX specifically we may take
the example code \ref[onnx_in].

\midinsert
\label[onnx_in]
\begtt \hisyntax{python}
class OnnxModule(nn.Module):
    def __init__(self):
        super().__init__()

    def forward(self, x):
        return x + 3
\endtt
\caption/f Python code to be converted to ONNX IR
\endinsert

Which may be seen by the ONNX exporter\fnote{Shortened by hand for readability} as \ref[onnx_pseudo].

\midinsert
\label[onnx_pseudo]
\begtt
class GraphModule(torch.nn.Module):
    def forward(self, x: "f32[100, 128]"):
         # File: ./onnx.py:48 in forward, code: return x + 3
        scalar_tensor_default:
          "f32[]" = torch.ops.aten
                         .scalar_tensor
                         .default(3, dtype = torch.float32)
        add:
          "f32[100, 128]" = torch.ops.aten
                                 .add.Tensor(x, scalar_tensor_default)
        return (add,)
\endtt
\caption/f ONNX Pseudocode
\endinsert

Before being compiled into a low-level representation as shown in \ref[onnx_ir].

\midinsert
\clabel[onnx_ir]{ONNX IR}
\begtt
graph(
    name=main_graph,
    inputs=(
        %"x"<FLOAT,[100,128]>
    ),
    outputs=(
        %"add"<FLOAT,[100,128]>
    ),
) {
    0 |  # node_Constant_0
         %"val_0"<?,?> <- ::Constant() {
         %  value=Tensor<INT64,[]>(array(3), name=None)
         %}
    1 |  # node_Cast_1
         %"scalar_tensor_default"<FLOAT,[]> <- ::Cast(%"val_0") {
         %  to=FLOAT
         %}
    2 |  # node_Add_2
         %"add"<FLOAT,[100,128]> <- ::Add(%"x", %"scalar_tensor_default")
    return %"add"<FLOAT,[100,128]>
}
\endtt
\caption/f ONNX IR
\endinsert

How this is done from arbitrary Python code is by utilising
a \code{FakeTensor} class,
on which the algorithm actually runs,
however instead of the operations
happening they are recorded in the background
and the result is again always a \code{FakeTensor}.
After the algorithm finishes,
the libraries inspect what operations were performed
on what tensors and constructs an equivalent program
through some backend, TorchScript, Dynamo, etc.
This approach allows quite arbitrary Python programs
to be compiled however at the loss of semantics.
For example the classic \code{map(lambda x: x + 2, list)}
would just be compiled into a series of completely unrelated
additions on some projected content of the tensor.

These models also always have a static type of input and outputs
that may be unconstrained by the Python class which implements it
but must be specified explicitly when exporting.
The part of the type that determines the size of a tensor
is usually called their "shape" in this context.
We set our input shape to \code{(100,128)} in this example
so the exported graph has annotations like \code{f32[100, 128]}.

Oftentimes a model is made to accept a large "batch" of inputs
to parallelize inference.
This is implemented as simply
setting the first dimension as dynamic,
resulting in shapes such as \code{(?,100,128)}.
Models must be explicitly resized before using this dynamic size
as it will default to 1 otherwise.

\midinsert
\begtt \hisyntax{python}
shape = (BATCH_SIZE, *shape[1:])
model.resize_tensor_input(0, shape)
model.allocate_tensors()
model.set_tensor(0, input_data)
\endtt
\caption/f Setting input tensors with batches
\endinsert

The subset of python supported is quite large,
even code such as \code{[ sum(n) for n in x ]}
will successfully compile into a graph,
the efficiency of which depends heavily on
the type of code,
for example the above generates a massive
array of round-robin addition nodes.
Of course calling external functions is also supported
and these are inlined and compiled as if written
directly inside forward.

If one wishes to have control over what
exactly their code compiles to,
it is better to directly use modules from the given library.

These can be initialized under \code{__init__}
as instance variables and then utilized in
the \code{forward} function like so:

\begtt \hisyntax{python}
def __init__(self):
  super(..., self).__init__()
  self.relu = nn.ReLU()

def forward(self, x):
  x = self.relu(x)
  return x
\endtt

It is also useful to note that this code defines
string names for the inputs and outputs of a graph,
these are then used to retreive many outputs or set inputs
in a human-readable way.

\secc \code{forward} (mandatory)

This is the crux of the pipeline,
it defines the actual operations performed on
the inputs and returns what the Graph node would have
as outputs.

If in need of multiple inputs and outputs
the function is allowed to take more than 1 argument,
and they will be matched according to name.
If on the other hand I want to return
multiple outputs we can wrap
the return in a dictionary with keys
being the names of our outputs and values being
the given output tensors.

\secc \code{__init__}

Some frameworks, namely ONNXRuntime want their modules
to be attributes of the given object,
and so we usually then initialise our submodules here and refer to them in \code{forward}.
For example including \code{self.conv1 = ...} inside the constructor
and then calling as \code{x = self.relu(x)} in our \code{forward} function.

\secc Movement functions

Methods such as \code{.to()}/\code{.cpu()}/\code{.ipu()},
cause our model to be run on the given target device.
However, this also hints to the given framework
what the target is which is also taken into account
in cases of optimization.

\sec Keras

TensorFlow interops and includes a complete interface to the Keras library,
which includes its own additional ways to define models.
For instance \code{keras.api.applications} contains prebuilt models
for various uses such as image recognition
that can then be freely used and manipulated for our needs.
Another way is the \code{keras.models.Sequential} class,
which takes a list of layers and concatenates them
one after another into a model.
Keras layers are much like TensorFlow modules.

\sec Other Tools

\secc Visualization: Netron \& Zetane 

Netron from \cite[NetronApp] is an incredibly
useful open-source tool for inspecting,
visualizing and generally debugging machine learning model graphs.
Of relevant formats it supports ONNX, TensorFlow and Keras,
along with support to display all metadata contained in a given layer.
It runs either as a web app or an offline local program.
Netron provides an interactive graph GUI
which allows searching, layouting, and
exporting to PNG or SVG.
An example of such an export is figure \ref[netron].

\midinsert
\label[netron]
\cinspic{./netron_test.pdf}
\caption/f Example of a Netron graph render
\endinsert

Zetane from \cite[Zetane] deserves an honourable mention,
it is a closed-source batteries included
environment for designing ML software and
includes a viewer very similar to 
Netron.
This viewer however is much
more capable and includes 3D visualizations
of convolutions on images, heatmaps,
graphs of values,
running the model with weights
and inspecting propagation.
It is also callable directly via a Python API,
so unlike Netron it can be integrated as a frontend into
other projects.

\secc Conversion: Tensor2onnx \& Tflite2onnx

All manner of programs exist to convert between
model formats,
which is not a trivial task however.
Very generally models can be converted
but some programs lose details,
such as weights, names,
structures that are concisely represented
may unwrap into larger forms, and more.
For a given application,
care should be given to verify
that all required information is intact.

\chap Hardware Specifics

\sec NPU Contents

This entire section references mainly \cite[IMx8mPlusAp].

\secc Parallel Processing Unit (PPU)

SIMD4, with 4 units with 256 threads each.

\secc Neural Network Engine

Does 1152 MAC operations per clock cycle.

\glos {MAC}{Multiply–accumulate operation $a \leftarrow a + (b \times c)$}

\secc Tensor Processor

Supported operations are

\ctable{ll}{
Pooling & Max, average \cr
Unpooling & Yes \cr
Activation & ReLU, Leaky ReLU (LUT for other types) \cr
Normalization & Yes \cr
Region Proposal Support & Yes \cr
}

\secc Interrupts

The NPU can send CPU interrupts, these are set using the driver and cannot in fact be understood without the driver since the NPU and driver both assign arbitrary meaning to the given bits of the interrupt info register.

\secc OpenVX Hardware Support
.\rfc{BIG TODO}

\sec Configuration Environment Variables read by TIM-VX

In the i.MX Machine Learning User's Guide \cite[IMxMachineLeLfRe2024] section 6.1.2,
we can read about configuration variables.

\begitems \style d
* {\code{USE_GPU_INFERENCE}} As the NPU and GPU share this driver,
TIM-VX uses the value of this variable to determine
which to use, "1" for GPU or "0" for NPU.
\enditems

\secc Profiling
\begitems \style d
* {\code{CNN_PERF}}

Prints how long operations take.
Requires \code{VIV_VX_DEBUG_LEVEL=1} and implied by \code{VIV_VX_PROFILE}.

* {\code{NN_EXT_SHOW_PERF}}

Shows the details of how the compiler determines performance.

* {\code{VIV_VX_PROFILE}}

Enables creation of \code{vprofiler_xxx.vpd} files which may be examined using the Vivante vAnalyzer tool from the Vivante VDK.
Information is either per-node (value: \code{"1"}) or per-graph (value: \code{"2"}).

* {\code{VIV_VX_DEBUG_LEVEL}}

Prints extra debug information.

* {\code{VIV_MEMORY_PROFILE}}

Only applies to CPU/GPU.
\enditems

\secc Model Caching
\begitems \style d
* {\code{VIV_VX_ENABLE_CACHE_GRAPH_BINARY}}

Enables saving of the compiled graph to disk with a hash
so that the next time the same graph would be compiled,
this \code{*.nb} file will be loaded instead.
In addition to this the documentations claims that
warmup time may take more than one inference.

* {\code{VIV_VX_CACHE_BINARY_GRAPH_DIR}}

Directory to save the cache files to.
\enditems

\sec Power Modes

The NPU can be set to 4 different states,
\begitems \style d
* {On} Standard full-power
* {Off} Can be powered off, since after leaving this state the device is reinitialized
* {Idle} Clock speed lowered to $1/64^{th}$
* {Suspend} Idle clock speed and requires some time to reach idle
\enditems
%% \rfc{Performance Counters for DMA Profiling}

%% \sec Loading into Memory
%% \secc Direct loading of images
%% \rfc{TODO}

\chap Frameworks

%% \sec Keras

\sec LiteRT (Lite RunTime)

This project is closely related to tensorflow
and acts as its ambassador for resource-constrained devices.
Tensorflow is used to train, prepare, and debug the model,
before it's deployed to a device
where only a stripped down runtime is installed.
As such it does not contain much beyond what is needed
to run inference.
Models need to be converted
into the FlatBuffers tflite format before use\cite[LitertOverview].

When running the library we must set an environment variable which \code{libvx} checks
to see which device it should use, either NPU or GPU:

\begtt
export USE_GPU_INFERENCE=0
\endtt

Next we must load the external dynamic library
which we pass into the LiteRT interpreter: \ref[litert_delegate].

\midinsert
\label[litert_delegate]
\begtt\hisyntax{python}
import tflite_runtime.interpreter as tflite

# load external delegate
external_delegates = [
    tflite.load_delegate("/usr/lib/libvx_delegate.so", "")
]
\endtt
\caption/f Loading LiteRT delegate
\endinsert

Delegates are shims between the Tensorflow python library
and the external system library acting as device driver,
so even though we want to use \code{libOpenVX.so},
we instead load the \code{libvx_delegate.so}
library which links against it.

\midinsert
\label[litert_inter]
\begtt\hisyntax{python}
interpreter = tflite.Interpreter(
  model_path=args.model_file,
  experimental_delegates=external_delegates,
)
\endtt
\caption/f Create the LiteRT interpreter object
\endinsert

After creating the interpreter object with \ref[litert_inter]
we can interact with it in one of two ways.
Either we have a prepared calling convention called a signature
inside the model and call that as in \ref[litert_signat]
which directly returns an output tensor.

\midinsert
\label[litert_signat]
\begtt \hisyntax{python}
signature = model.get_signature_runner()
signature(x=<tensor>)
\endtt
\caption/f LiteRT signature runner
\endinsert

Or we go about explicitly setting and reading
each input and output.
For this it is useful to know their properties,
so we may call \code{model.get_input_details()}
or \code{model.get_output_details()} respectively.
These functions return lists of detail objects
containing the shape, name and index.
This index can be used to read or write tensors with \code{set_tensor} in \ref[litert_set_tensor].

\midinsert
\label[litert_set_tensor]
\begtt \hisyntax{python}
model.set_tensor(model.get_input_details()[0]['index'], input_data)
model.get_tensor(model.get_output_details()[0]['index'])
\endtt
\caption/f LiteRT directly setting input tensors
\endinsert

\sec Tensor Virtual Machine (TVM)

As the full name Apache TVM\cite[ExtendingTensoWang2022] might suggest,
this is an open-source framework built by Apache
in a similar vein to ONNX including their own
IR called Relay that is supposed to allow
optimizations to be shared between multiple target backends.

It does not have its own format for files,
opting to be able to import any of the other frameworks'.
It does have the additional capability to compile
down models into C++ libraries in the form
of shared object files.

According to the recipe's \code{EXTRA_OECMAKE} and the fact that it links against tim-vx
in the tvm bitbake recipe,
support for VSINPU should already be built in.

The \code{tvm.relay.frontend} module allows us to load
a model, from almost any format user in machine learning,
using functions such as \code{from_keras()}, \code{from_onnx()},
\code{from_paddle()} or \code{from_tflite()}.

However merely trying to import the \code{tvm.relay} module
throws an error concerning python not being able to find
the scipy library.
Scipy is not packaged by any of the standard NXP layers
and didn't have a workable recipe that could be found
before this went out of scope for this work.

Another interesting property of TVM is that
ONNX can use it as a backend directly.
Strangely enough ONNX even when compiled with
\code{-Donnxruntime_USE_TVM=ON}
does not acknowledge TVM as a valid backend.

\sec ONNX (Open Neural Network Exchange)

Though originally authored by joint efforts of Facebook and Microsoft,
this project has flourished into a widely supported open-source ecosystem
\cite[WikiONNX, ONNX].
It comprises a comprehensive specification for models, formats, types, operators and
abstract data descriptions.
It includes protobuf definitions of their \code{.onnx} model files.
It used to support only inference,
but training was added with ONNX IR spec version 7\cite[OpenNeuralNet].
Models here are represented as either just a stateless
inference function in the case of inference-only models,
or may be extended with an initialization and a training
method which may modify internal stateful variables
of the given model.
ONNX also encodes a block of operators that encode a more complex task,
these may be substituted for builtins by the runtime based off
of their name\fnote{Went through changes in IRv9}.
The internal structure is a list of acyclically dependent topologically sorted nodes,
each node providing name, metadata, i/o and the given operation performed.
ONNX however does provide HOF-like operators that are applied
to entire subgraphs,
thus substituting the need for self-references\cite[OnnxConcepts].
These nodes are strung together as a pipeline each input being connected
to a previously declared output of the same name.
Each output name is unique since SSA is mandatory,
verification tools for this and other properties are
available from the creators of ONNX.
The last argument of some operators is marked as variadic,
thus allowing as is traditional with
regular languages to pass an arbitrary number of
inputs/outputs to it,
obviously respecting the minimum arity.
%% \rfc{ONNX vs. ONNX-ML}
These graphs are meant to be assembled programatically,
however ONNX does provide a textual form of their files.

\secc NPU backend

Implementing a backend for ONNX is
done by providing a Python shim wrapping the functionality
as we have compiled in support for VSINPU,
we can simply list it as our inference session provider
and run inference as is shown in Snippet \ref[onnx_infer].

If ONNXruntime is downloaded from pip,
we can use the available provider
CPU and Azure.
VSINPU is listed as known but unavailable.
Activating it yields an unavailable error
so the ONNX we use must be from Yocto.

\midinsert
\label[onnx_infer]
\begtt\hisyntax{python}
import onnxruntime
session = onnxruntime.InferenceSession(
  "model.onnx",
  providers = [ "VSINPUExecutionProvider" ]
)
session.run(
  input_feed= {
    "x": np.full(
           fill_value=[1.0],
           shape=(1,28,28),
           dtype=np.float32,
         )
  },
  output_names = ["add"],
)
outputs = session.run(None, {"input": inputTensor})
\endtt
\caption/f Running on NPU
\endinsert

You may convert any torch model into ONNX.
Snippet \ref[torch2onnx] shows how
given a torch/tensorflow model one can export
it into a onnx file.

\midinsert
\label[torch2onnx]
\begtt\hisyntax{python}
torch.onnx.export(
  model, # model being run
  torch.randn(1, 28, 28), # model input (or a tuple for multiple inputs)
  "fashion_mnist_model.onnx", # where to save the model
  input_names = ['input'], # the model's input names
  output_names = ['output'], # the model's output names
)
\endtt
\caption/f Exporting Torch to ONNX
\endinsert

\sec OpenCV
\rfc{BIG TODO}
\begtt
cv2.error: OpenCV(4.10.0) /usr/src/debug/opencv/4.10.0.imx/modules/dnn/src/onnx/onnx_importer.cpp:1057: error: (-2:Unspecified error) in function 'handleNode'
> Node [Conv@ai.onnx]:(onnx_node!node_Conv_0) parse error: OpenCV(4.10.0) /usr/src/debug/opencv/4.10.0.imx/modules/dnn/src/layers/layers_common.cpp:106: error: (-5:Bad argument) kernel_size (or kernel_h and kernel_w) not specified in function 'getKernelSize
\endtt

\begtt\hisyntax{python}
import cv2 as cv
m = cv.dnn.readNet('model.onnx')
m.setPreferable
m.setPreferableBackend(cv.dnn.DNN_BACKEND_TIMVX)
m.setPreferableTarget(cv.dnn.DNN_TARGET_NPU)
import numpy as np
m.setInput(np.full(fill_value=[1.0], shape=(1,28,28)))
m.forward(["output"])
\endtt

Yields:
\begtt
Traceback (most recent call last):
  File "<stdin>", line 1, in <module>
cv2.error: OpenCV(4.10.0) /usr/src/debug/opencv/4.10.0.imx/modules/dnn/src/net_impl.cpp:279: error: (-204:Requested object was not found) Layer with requested id=-1 not found in function 'getLayerData'
\endtt
Which is the error for wrong output name.

\code{m.forward([])} and \code{m.forward()}
seem to both yield the same array,
seems that if no outputs are specified,
then implicitly it takes all.

\begtt
>>> m.getInputDetails()
Traceback (most recent call last):
  File "<stdin>", line 1, in <module>
cv2.error: OpenCV(4.10.0) /usr/src/debug/opencv/4.10.0.imx/modules/dnn/src/net_quantization.cpp:268: error: (-6:Unknown error code -6) Net isn't quantized in function 'getInputDetails'
\endtt

\sec Unaddressed frameworks

\secc The NXP eIQ environment

The NXP eIQ stack officially supports, LiteRT, ONNX, PyTorch and OpenCV.
Of these only LiteRT officially supports the included NPU.
Due to the fact that it only calls the other libraries listed
and that the environment itself
is a large and unwieldy project,
we opted to skip it in this work.

\secc Android NNAPI

Android provides a direct official API
for use in Machine Learning acceleration\cite[AndroidNNAPI].
It is however only exclusively available on Android
which requires negotiating with NXP
for an image and is to be deprecated in the future
in favour of LiteRT.
The Linux Kernel also has something called the
NNAPI, however that is completely unrelated.

\secc Paddle Paddle
\rfc{BIG TODO}

\sec \OpenVX~

\OpenVX~is a standard hardware API that can be
implemented by hardware vendors of
hardware accelerators
and exposed as a C API to users \cite[OpenvxPortab2011].
The Khronos group that
manages the \OpenVX~specification
only describes an abstract machine
and certain operators with defined semantics.
The implementation of the operators
in the means of hardware
is up to the vendor,
and the specification aims to be written
in a way as to allow as much optimization
as possible.
When we spoke so far of \code{libOpenVX.so},
that is \VeriSilicon{}'s pre-compiled
implementation of the \OpenVX~API.
\OpenVX~officially aims at vision processing specifically,
however there are Neural Network extensions to allow
using the same pipeline to also accelerate
machine learning operations utilizing the tensor
structure from the full \OpenVX~v1.2 Spec,
and extending it with new operators such as
\code{vxActivationLayer},
\code{vxConvolutionLayer},
\code{vxFullyConnectedLayer},
\code{vxSoftmaxLayer} and
more\cite[OpenVXNNE].

Known Implementors of this API are:

\begitems \style d
* {VeriSilicon}

The \code{libOpenVX.so} object all our libraries link against.

* {KhronosGroup/OpenVX-sample-impl}

Only truly open-source implementation,
however it is supposedly very ad-hoc,
slowly implemented,
and emulates the given operators on
CPU or related using OpenCL \cite[KhronosgroupOp2024].

* {TexasInstruments/tiovx}

Source available but only authorized for use on TI hardware
\enditems

However only the relevant vendor makes sense
to use as it is tightly coupled to the given device
that is to be controlled with it.

\OpenVX~still uses a graph abstraction,
however the operations are far too low-level
to be convenient and useful for the purposes
of general purpose machine learning jobs.
Especially since they do not
interoperate with existing well known
formats from the public machine learning ecosystem.
Which is why we opted to rather
focus on relevant and convenient libraries
that wrap \OpenVX~instead of calling it directly.

\chap Speedup
\sec Theoretical advantages
\sec Benchmarking Practical Results

Running the same model under LiteRT and ONNX,
yields interestingly different results for CPU
speed and NPU Warmup, which makes sense
as even though the models implement the same
structure they have different representations and
so may be compiled very differently.
Seems that once the model is up and running though
the NPU has the same performance under both.

\midinsert
\clabel[perf]{Performance Metrics}
\ctable{lrrr}{
Framework & CPU & NPU Warmup & NPU \crl
ONNX & 50ms & 16ms & 4ms \cr
LiteRT & 132ms & 380ms & 3ms
}
\caption/t mobilenet_v1_1.0_224_quant Performance Metrics
\endinsert

The NPU is optimized for specific tasks
so certain operands may end up just being run
on the CPU even if the NPU is set as the target,
only emitting a Warning notice to stderr.

In addition, an atypical model may even run slower
on NPU due to the various overheads if
it is not something the machinery knows how to handle,
bumping a 52ms runtime on the CPU up to
a consistent 200ms on the NPU.
I reiterate,
enabling hardware acceleration slowed down
our inference to 25\% of what it achieved on just CPU.

\sec Device Tensors
.\urlnote{https://onnxruntime.ai/docs/performance/device-tensor.html}

\sec Startup slowdowns

This is apparent when running the C++ example where
measuring the entire program's runtime
results in 0.2 seconds on CPU and
3.4 seconds when running on NPU
even though the internal timer
measuring pure inference time shows
that NPU speeds up inference from 35ms to 3ms.

\secc Dynamic loading of Libraries

In the case of both the \code{Python} and \code{C++} interfaces,
using either \code{ONNX} or \code{LiteRT},
the case always is that the interfacing library
must be loaded dynamically.
Loading may take up a significant amount of time,
in one experiment the Python profiler
shows 7\% of the runtime taken up by \code{do_lookup_x} from \code{ld-linux.so}.

\secc Tiling and Compilation
.\rfc{BIG TODO}

\secc Bus

Since the NPU acts as a completely separate device from the
standard SOC's CPU and memory,
even having its own clock,
we must transfer the model and input parameters
over a AXI/AHB bus
and that takes time.
Behind that Bus lies a Memory Controller,
scheduler and more which all work together to slow down the call
especially initially.

\glos {AXI}{Advanced eXtensible Interface}
\glos {AHB}{Advanced High-performance Bus}

\secc NPU Clogging

During tests it often happens that if the input data
is too large, or the NPU is otherwise mishandled
then all further requests to it simply
block forever.
I am sure the NPU can be reset,
however we opted to just restart the device for now
whenever this occurs,
sometimes having to hard-kill all processes that
are working with the NPU as they even block system restarts.

\chap Suggesting PikeOS integration

We see that the most integral part of
porting this ecosystem to a given OS,
would be to get TIM-VX working as
that is targeted by all the other
libraries.
The hard part will be porting libOpenVX
as that is closed-source and shipped as a prebuilt
binary artefact.
This shared object binary can of course be patched
and edited to work around some issues,
however there is little that can be done
if the library does not work.
Luckily libOpenVX has very few runtime dependencies,
apart from the standard set of linux libraries
it requires libVSC, libGAL, libArchModelSw and libNNArchPerf,
all of which are also shipped as binary blobs.
Only shipping libOpenVX does not lessen the load substantially,
as all of these dependencies are also required by it.

Once that is in place TVM and OpenCV support extrernal
file types,
while ONNX, Keras and TensorFlow
provide tooling to convert and
often have built-in facilities to work with,
import and export
all of their respective formats.
And so supporting one of these well should
suffice for most applications.

All our mentioned frameworks have multiple sets of bindings,
primarily always Python and C++, along with a mix of other languages.
For performance sensitive applications it would be beneficial
to support the C++ API while Python will depend on
whether clients wish to prototype their applications
on the device or if the workflow
of training on external hardware and only running on the device
is sufficient.

\chap Conclusion

We set out to perform reconaissance of
hardware acceleration
in the context of machine learning on embedded devices.
Much of the author's time was spent
on familiarization with the bitbake Yocto build system,
BSPs,
flashing to the EVK,
navigating hardware vendor documentation,
and other themes completely tangential
to the problems we set out to address.
Despite that, we identified the primary
point of ingress for controlling the NPU,
several libraries that support it
even outside the NXP specification.

We either compiled these libraries with NPU support,
recorded what steps were done to do so
and afterwards described how to specify to the library
that it should utilize the NPU.
Or in the cases where a roadblock was hit
we described what the next step would
be in getting it to work.

The libraries were benchmarked on CPU and NPU,
concluding that in terms of raw NPU
inference perforance on the same model
their differences are negligible.

Finally we outlined which libraries are relevant
for PikeOS to claim NPU support
and briefly discussed from an outside perspective
how that might be done 

\glos {EVK}{Evaluation Kit}

\app Glossary
\glos {NBG}{Network Binary Graph}
\glos {VsiNPU}{\VeriSilicon~NPU}
\glos {SSA}{Single Static Assignment}
\glos {Tensor}{Sufficiently described for purposes of this text as multidimensional arrays}
\glos {IR}{Intermediate Representation}

\makeglos

\bibchap
\usebib/c (iso690) thesis

\app Build configuration
\label[buildconf]

\begtt
Build Configuration:
BB_VERSION           = "2.8.0"
BUILD_SYS            = "x86_64-linux"
NATIVELSBSTRING      = "universal"
TARGET_SYS           = "aarch64-poky-linux"
MACHINE              = "tqma8mpxl-mba8mpxl"
DISTRO               = "fsl-imx-wayland"
DISTRO_VERSION       = "6.6-scarthgap"
TUNE_FEATURES        = "aarch64 armv8a crc crypto"
TARGET_FPU           = ""
meta
meta-poky            = "HEAD:200d12b6a58ad961d60a7774ca0f7a9d29498724"
meta-oe
meta-python
meta-multimedia      = "HEAD:72018ca1b1a471226917e8246e8bbf9a374ccf97"
meta-freescale       = "HEAD:0627128b341cfb2bef7a0832ce8cac0ce1127f13"
meta-qt6             = "HEAD:586a6cb5aec755803a3be3cec359baafe89d6432"
meta-tq              = "HEAD:257b8c0b4b6df3bb27fb69bd2312dd254c73fed3"
meta-imx-ml
meta-imx-sdk
meta-imx-bsp         = "HEAD:219f6d04a4c339eb6f2dc626f944bbdf9a716ff5"
meta-arm
meta-arm-toolchain   = "HEAD:950a4afce46a359def2958bd9ae33fc08ff9bb0d"
meta-freescale-distro = "HEAD:b9d6a5d9931922558046d230c1f5f4ef6ee72345"
meta-overlay         = "<unknown>:<unknown>"
meta-virtualization  = "HEAD:6f3c1d8f90947408a6587be222fec575a1ca5195"
meta-filesystems
meta-networking      = "HEAD:72018ca1b1a471226917e8246e8bbf9a374ccf97"
meta-tpm
meta-parsec          = "HEAD:459d837338ca230254baa2994f870bf6eb9d0139"
meta-clang           = "HEAD:2b7433611d80f6d0ee1b04156fa91fc73d3c2665"
\endtt

\bye
